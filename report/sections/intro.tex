\subsection{Background}
\label{sec:intro:background}
Our premise is that user-vote based recommender systems such as those used by Netflix are not very successful in
providing users with new and previously unknown content. For example:

\subsection{Content-based recommenders}
These systems recommend items to a person similar to previous items rated highly by
that person will never recommend items outside of a person’s content profile.

\subsection{Collaborative Filtering}
These approaches where similar-voting users (nearest neighbors) are found in order to predict a
user’s vote for an unseen item are also limited; it cannot recommend new items as no-one has voted on them yet.
Besides that, voting based systems are inherently biased towards items that were highly rated by the mass of
users, this results in recomendations for overall highly rated movies and not for a personal recomendation.

\subsection{Vote-based recommending}
This is limited because there will always be a gap between a users’s discovered content
profile and their true content profile. For example in the case of movies:
\begin{itemize}
	\item Users will not vote on all movies that they watch.
	\item A recommender system has no knowledge from a user’s history unless it specifically asks the user to rate
movies the user has already seen. Otherwise the profile will only consist of movies that the user watched
since the user started using the system.
	\item A content provider cannot offer all possible content; therefore a recommender system working on that
content is limited because various items that distinguish a user’s “taste” from other users might not be
available and thus cannot be voted on. Content providers generally do not want to remind users that
content exists which is not available on their system.
\end{itemize}


\subsection{Goals}
\label{sec:intro:goals}
A recommender system is only truly useful if it recommends items that are unknown to the user; therefore our
objective is to develop a novel approach that results in a recommendation system that excels at recommending items
that the user did not yet know; we will specifically focus on movies in this project.

\subsection{Solution}
\label{sec:intro:solution}
Our solution to the problem is to use a social network based approach. If all movies are vertices in a graph, then we
can create edges between those movies if they share actors, directors, writers, producers, involved companies or other
connections. This edges will have a different weights based on the type of connection they have. We can obtain this
data by mining various public internet sources. Just as a social network “seeds” your account by importing contacts,
we can seed this movie network automatically or manually (by the user). Movies can be recommended by doing graph
searches, calculating distance in number of nodes of known-nodes to unknown-nodes or even by doing random jumps
in the graph. We can also mine movie review- and discussion websites to find movies that are often mentioned
together and create direct edges between those movies. We can also let users do graph searches by themselves, similar
to Facebook’s social graph search. And because we don’t take voting into account, we expect this approach to work
better regarding the finding of new movies that a user may be interested in.