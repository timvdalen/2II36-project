\subsection{Data}
For the trakt data, we created a program that used the Trakt API to gather the IMDb IDs for each title.
The program was online for three weeks. 
In these three weeks it looked up the ID of more than 200.000 movies. 
However for only 7255 movies the program found the ID.
We then realised we could search each movie before the deadline and decided to find the ID's of the metacritic movie.
After we found the ID's for these movies, we wanted to gather the comments for these movie with the same program.
Unfortunately the API went offline and thus we were enable to gather the Trakt comments. \\

IMDb generates plain text files with all content on their website.
This plain text file were really hard to parse.
The plain text file contained a lot of illegal characters and the structure of the file was not constitent. 
IMDb also contains a lot of movies. 
A lot of movies are not worth recommending to the users.
Most movies in other languages than English are not fun for users to watch. \\

We got all the user reviews of Metacritic. For most popular movies there were enough user reviews.
For movies that are not very popular, there were not much user reviews.
This means that for some movies we could not find mentions to other movies in the Metacritic data.

\subsection{Mentions}
To find mentions we searched for the exact title of movies. 
This means we had some problems with the found mentions.
We had a hard problem finding out if a mention was really, because they recommend that movie
This occurs with movies with really short names for example: X, 9. 
Or with movies names, that are also English words like: Yes, No, Take.
It also happened with movies that are part of another movie. 
When people in the user reviews of Batman refer to Man of Steel, they also mention the movie Steel.
This happens for a lot of movies. 
For example The Ring and Lord Of The Rings, Dragon and How to train your dragon. 
Another problem that happend a lot was when there are movies with the same name of a popular actor.
Peter Parker plays in Spider-man and is mentioned a lot in the user reviews. 
There also is a movie Parker and now spider-man is linked to Parker.
Perhaps we could solve this, by dleleting all mentions where an actor in that movie has the same name as the mentioned movie. \\

Another problem with the mentions is that we do not know if they mention the movie in a positive way. 
The user review could be very postive about the movie and say movie X is way better than this movie Y.
This means that we would give a bad recommendation instead of a good one. 

\subsection{PathSim}
Using the IMDb data and the �mentions� data that was mined, we can construct a graph. This
graph can be used to find recommendations, but to find these recommendations we need to have
a measure of similarity between nodes.

The similarity measure that is currently used to generate the view on our website is purely based
on mentions; given a node, we can give a list of other nodes based on the mentions relations. If
there are only a few mentioned nodes, we can use mentions-of-mentions to expand the result, but
this is often not necessary.

Another similarity measure that is more similar to PathSim is used in the command-line tool that
was demoed but which is not available online. In that second variant a query is run which given two
nodes, outputs movies of actors that played in both given movie nodes. To put this in PathSim
terms, we define a meta-path movie-actor-movie to define movie similarity.

With PathSim, however, you count the number of paths between nodes X and Y following a certain
meta-path. Instead, we we take one path following the meta-path between two given nodes and
find movies that that actor played in (meta-path actor-movie).

An alternative variant that we could not test because it was too computationally expensive used the
meta-path movie-actor-movie-actor-movie, where the two �outer movies� are user-determined and
we try to find all paths that match the meta path. This query turned out to be too expensive for
Neo4j, and our currently query-pipeline can only support queries running within interactive
timescales. To run queries of this scale, we would need an infrastructure that runs long-term jobs
and can store the result in a cache-like database where it can be used by the front-end. This would
be an area of future work.
