We have used the following data sources.

\subsection{Metacritic}
\label{sec:data:metacritic}
Metacritic collects reviews from other websites for movies, books, computer games, albums and TV shows. 
They combine the review grades from each reviews to derive a general grade for the movie.
We are only interested in their data about movies, but of course they also do this for books, computer games, albums and TV shows.
They also let registered users write reviews about the movie and combine their grades for a user grade for the movie. 
Unfortunately there is no official Metacritic API. There however is an unofficial Metacritic API. 
This API can be used to find grades and information about the movie.
We are interested in the critic reviews and user reviews, but that information is not accessible through the API, thus we had to build our own web scrapper to collect the data of Metacritic. 
Metacritic only displays the first part of each critic review on their own website. 
The rest of the review is available on the website of the writer of the review. 
There are a lot of different writers of critic reviews, thus it would be extremely complex to automatically scrap their websites correctly. 

So we were going to focus on the user reviews. These are displayed on a different page for each movie. 
We decided to scrap the first 100 helpful user reviews of each movie. 
The next problem was to get the url of those movies. 
Metacritic only has around 7000 movies, while the IMDB list contained 1.4 million movies. 
So we just scrapped all the movies from Metacritic and searched for their IMDB ID’s after that. 

Now we knew what data we had to collect we could build a scrapper.
The scrapper works in two phases:
\subsubsection{Collect movie URL’s }
Metacritic has a list of all movies by their first letter. This list contains the url’s to the page of the movies. 
There are up to hundred movies per page, so we build a scrapper that downloads these pages. 
The url’s were easily extracted from the downloaded pages, by selecting the main content of the pages and selecting all links to other pages. 
Then we filter out the links that are not movies. These links do not have movies in the url and thus were easily filtered. 
\subsubsection{Collect user reviews}
For each movie we downloaded the page that consisted the user-reviews, with up to hundred reviews sorted on most helpful. 
We extracted the user reviews from these pages by look for the css classes with div.review body span. 
There was a little problem when user reviews where to long. These user reviews were three times in the page. One short version, two long versions. 
We only needed the last long version. So in each class with  div.review body span, 
we checked if it ended with expand… and if this was true we skipped that class and the next and only used the last class. 
\subsubsection{Crawling issues}
To deal with politness we crawled the metacritic pages single threaded. 
We also crawled at most 3000 pages a day. 
This meant we did not have to deal with scheduling the pages. 
We just visited each url by alphabetical order. 

We did have a small problem with redirecting url's . 
The crawler could not handle those pages. 
Thus we translated those url's to the real url's. 
\subsection{trakt.tv}
\label{sec:data:trakt}
Trakt is a web service that lets people track what TV shows and movies they have watched.
It is known for its well-structured - though unfortunately not so highly available - API.

Trakt gets data about movies and TV shows from TMDb, a site that scrapes IMDb and makes movie data easier available.
Users of Trakt can leave reviews and comments (called shouts) on movies.

\begin{figure}[H]
	\centering
	\includegraphics[width=\linewidth]{sections/data/trakt}
	\caption{Trakt data sources}
\end{figure}

We use the Trakt API to get all reviews and shouts for movies that we are interested in.

\subsection{Reddit}
\label{sec:data:reddit}
Reddit is a large online community and news website where registered users can submit content, such as text posts or direct website (perma)links. 
Registered users can vote submissions \textbf{up} or \textbf{down} to organize the posts (and comments) and determine their position on the site's pages. 

\begin{figure}[H]
	\centering
	\includegraphics[width=\linewidth]{sections/data/reddit}
	\caption{Reddit data source}
\end{figure}

Reddit is divided into sections called \textbf{subreddits}. 
Within these subreddits users can discuss various kinds of topics related to that subreddit.  

\subsubsection{Find movie discussions}
Reddit offers an advanced \textbf{API} that allows for searching. This includes searching for matches in the titles of a topics. By searching for movie titles in these topic titles we are able to find topics that might be discus this specific movie and thus providing us with useful movie information. We will restrict our topic searches to a subreddit called \textbf{/r/movies}; this section covers all kinds of news and discussions about movie, giving us more confidence that topics matching the search will actually cover a movie. Movie titles can be very generic and would give us topic matches that have nothing to do with the movie in question. We use the following query to find related topics.
\\
\\
\textbf{Query:} \emph{subreddit:\textbf{movies} title:\textbf{'movietitle'}}

\subsubsection{Crawling comments}
For every matching topic we use the API to retrieve all the corresponding comments (including nested ones). 
This gives between 0 and 1000 comments for every movie in our database.

\subsection{Mention extraction}
For every movie with comments we run a mention extractor in order to find mentions of other movies or actors within these comments.
We first combine all comments to one large comment for that movie. 
After that we scan each comment for each movie title and count the number of times it is mentioned.
When then know for each movie how often other movies are mentioned.
These are stored as relationships between movies.
We gave those relationships two properties. 
The type of the relationship, which is mentions for all relationships and the weight, which is the number of times the movie is mentioned.

We have chosen not to run a sentiment analysis on these comments. 
We do not know if the comment is negative about the movie or about the mention.
Suppose we have two sentences from comments on movie A: 
1. If you enjoyed this movie you should watch movie B which is really good. 
2. This movie was really good, way better than movie B.
Sentence one is positive and the mention is positive too, but sentence two was positive too, but the mention is negative. 
Because we do not know if the mentions are positive or negative by a simple sentiment analysis we decided to ignore sentiment. 